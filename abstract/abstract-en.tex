%-----------------------------------------------
% Template para criação de resumos de projectos/dissertação
% jlopes AT fe.up.pt,   Fri Jul  3 11:08:59 2009
%-----------------------------------------------

\documentclass[9pt,a4paper]{extarticle}

%% English version: comment first, uncomment second
\usepackage[english]{babel}     % English
\usepackage{graphicx}           % images .png or .pdf w/ pdflatex OR .eps w/ latex
\usepackage{times}              % use Times type-1 fonts
\usepackage[utf8]{inputenc}     % 8 bits using UTF-8
\usepackage{url}                % URLs
\usepackage{multicol}           % twocolumn, etc
\usepackage{float}              % improve figures & tables floating
\usepackage[tableposition=top]{caption} % captions
%\usepackage{parskip}

%% page layout
\usepackage[a4paper,margin=30mm,noheadfoot]{geometry}

%% space between columns
\columnsep 12mm

%% headers & footers
\pagestyle{empty}

%% figure & table caption
\captionsetup{figurename=Fig.,tablename=Tab.,labelsep=endash,font=bf,skip=.5\baselineskip}

%% heading
\makeatletter
\renewcommand*{\@seccntformat}[1]{%
  \csname the#1\endcsname.\quad
}
\makeatother

%% avoid widows and orphans
\clubpenalty=300
\widowpenalty=300

%% macros
\newcommand{\evm}{Eulerian Video Magnification}

\begin{document}

\title{\vspace*{-8mm}\textbf{\textsc{Android-based implementation of Eulerian Video Magnification\\for vital signs monitoring}}}
\author{\emph{Pedro Boloto Chambino}\\[2mm]
\small{Dissertation supervised by \emph{Prof.\ Luís Teixeira} and \emph{Luís Rosado}}\\
\small{at \emph{Fraunhofer Portugal AICOS}}}
\date{}
\maketitle
%no page number
\thispagestyle{empty}

\vspace*{-4mm}\noindent\rule{\textwidth}{0.4pt}\vspace*{4mm}

\begin{multicols}{2}

\section{Motivation}\label{sec:motivation}

\evm{} is a method, recently presented at
\emph{SIGGRAPH}\footnote{\url{http://www.siggraph.org/}} 2012, capable of
revealing temporal variations in videos that are impossible to see
with the naked eye. Using this method, it is possible to visualize
the flow of blood as it fills the face~\cite{Wu2012Eulerian}.
And to assess the heart rate in a contact-free way using a
camera~\cite{Wu2012Eulerian, Poh2010Non, Poh2011Advancements}.

There has been some successful effort on the assessment of vital
signs, such as, heart rate, and breathing rate, in a contact-free
way using a webcamera~\cite{Wu2012Eulerian, Poh2010Non, Poh2011Advancements},
and even a smartphone~\cite{Vitrox2013, Philips2013}.

Since it is a cheap method of assessing vital signs in a
contact-free way, this research work has
potential for advancing fields, such as, \emph{telemedicine},
\emph{personal health-care}, and \emph{ambient assisting living}.

Despite the existence of very similar products by
\emph{Philips}~\cite{Philips2013} and
\emph{ViTrox Technologies}~\cite{Vitrox2013}
to the one proposed on this research work, none of these implement
the \evm{} method.

Due to being recently proposed, the \evm{} method implementation
has not been tested in smartphones yet.

\section{Objectives}\label{sec:objectives}

The main goal is to develop a lightweight, real-time \evm{}-based
method capable of executing on a mobile device. Which will require
performance optimizations and trade-offs will have to taken into account.

In the process, the creation of an Android application which
estimates a person's heart rate in real-time using the device's camera
will be developed.

\section{Work description}\label{sec:work}

In order to create an Android application capable of estimating a person's
heart rate, a desktop test application was first developed because of
its faster implementation speed, and easier testing.

The application was written in C/C++ with the aid of OpenCV library, an
open-source image processing library. Hence, the integration with the Android
platform was done through the Android Native Development Kit and the
Java Native Interface framework.

The application workflow started by grabbing an image from the device's
camera. A person's face was detected using the OpenCV object detect module
which was previously trained to detect human faces. A region of interest (ROI)
of the person's face would then be fed into the implemented \evm{} method to
amplify color variations. The average of the ROI green channel was computed,
in order to increase the signal-to-noise ratio, and stored. Along the time,
these stored values represent a
photo-plethysmographic~\cite{Verkruysse2008Remote} signal of the underlying
blood flow variations. The signal is further processed using the detrend
method~\cite{Tarvainen2002Advanced}
to remove trends from the signal without magnitude distortion. It is then
validated as a cardiac pulse signal by detecting its peaks in order to
verify its shape and timing. Finally, the heart rate estimation is
computed by identifying the frequency with the higher power spectrum of
the signal.

\subsection{Eulerian Video Magnification}\label{sec:work:evm}

This work initial implementations of the \evm{} method were written in Java.
However, because OpenCV for Java support was still recent and due to performance
reasons the final implementation was written in C/C++.

The \evm{} method approach combines
spatial and temporal processing to emphasize subtle temporal changes
in a video. First, the video sequence is decomposed into different
spatial frequency bands. Because they may exhibit different
signal-to-noise ratios, they may be magnified differently.
In the general case, the full Laplacian pyramid
may be computed. Then, temporal processing is performed on each
spatial band. The temporal processing is uniform for all spatial
bands, and for all pixels within each band. After that, the extracted
bandpass signal is magnified by a factor of $\alpha$, which can be
specified by the user, and may be attenuated automatically. Finally,
the magnified signal is added to the original image and the spatial pyramid
collapsed to obtain the final output.

In this work, amplification of color variation was more important than
motion magnification. Thus, the spatial filter used was the computation
of a Gaussian pyramid level, which consists of applying a Gaussian blur filter
and downsampling the image multiple times. However, because performance
was a priority and computing a Gaussian pyramid level was computationally
expensive, this step was changed to a single resize operation using the
interpolation method \emph{AREA} of the OpenCV library, which produces a
similar result to the computation of a Gaussian pyramid level.

Temporal filtering is used to extract the motions or signals to be amplified.
Thus, the filter choice is application dependent. For motion magnification,
a broad bandpass filter, such as, the butterworth filter, is preferred. A
narrow bandpass filter produces a more noise-free result for color
amplification of blood flow. An ideal bandpass filter is used
on~\cite{Wu2012Eulerian} due to its sharp cutoff frequencies. Alternatively,
for a real-time implementation low-order IIR filters can be useful for both:
color amplification and motion magnification. Therefore, the subtraction
of two low-order IIR filters were used in the \evm{} method implemented.

\subsection{Heart rate estimation}\label{sec:work:heart}

The heart rate estimation is computed using the \emph{Fourier transform}, which
is a mathematical transform capable of converting
a function of time, $f(t)$, into a new function representing the frequency
domain of the original function.

To calculate the power spectrum, the resulting function from the
\emph{Fourier transform} is then multiplied by itself.

Since the values are captured from a video, sequence of frames, the function
of time is actually discrete, with a frequency rate equal to the video frame
rate, $FPS$.

The \emph{index}, $i$, corresponding to the maximum of the power spectrum can
then be converted into a frequency value, $F$, using the equation:

\begin{equation}
  F = \frac{i * FPS}{2 N}
\end{equation}

where $N$ is the size of the signal extracted. $F$ can then be multiplied by
$60$ to convert it to beats per minute, and have an estimation of the heart
rate from the extracted signal.

\section{Conclusions}\label{sec:conclusions}

An Android application, named Pulse, which was capable
of estimating a person's heart rate using the \evm{} method was developed.

A lot of effort went into improving the application performance,
so it was able to execute on an Android device. Having a performance
increase of 22\% from the initial implementation, as suggested
by~\cite{Wu2012Eulerian}, to a performance optimized version.

Because of this the application heart rate estimations accuracy was low.
The Pulse application measurements from 9 participants were compared
to the readings of a sphygmomanometer. The agreement between the two
according to the Bland-Altman plots analysis had a mean bias of $-12.00$ with
95\% limits of agreement $-33.13$ to $9.13$ bpm.

\subsection{Future work}\label{sec:future}

Having developed a lightweight, real-time \evm{}-based method for the
Android platform which goal is to amplify color variations, the performance
of different variants of the \evm{} method could be improved. This would
increase the usage of this method in other kinds of devices and in other
kinds of applications.

Other uses for the \evm{} method could be studied, such as, using it as a
security camera to detect small motion by magnifying such motion, or to identify
suspicious people by detecting its heart rate in a contact-free way. Another
idea would be to use the \evm{} method with the objective of identifying
if a person is drunk or not, based on the work of~\cite{Koukiou2012Drunk}.

Nevertheless, the implemented Pulse application needs to improve its heart rate
estimations accuracy, and its face detection module in order to not lose track
of a person's face. In addition, other vital signs could be monitored, such as,
breathing rate.

\bibliographystyle{unsrt}  % numeric, unsorted refs
\bibliography{../myrefs}

\end{multicols}

\end{document}
